\documentclass{jss}

\title{FlexMix: A General Framework for Finite Mixture
  Models and Latent Class Regression in \proglang{R}}
\Plaintitle{FlexMix: A General Framework for Finite Mixture
  Models and Latent Class Regression in R}
\Shorttitle{FlexMix: Finite Mixture Models in \proglang{R}}

\author{Friedrich Leisch\\Technische Universit{\"a}t Wien}
\Plainauthor{Friedrich Leisch}

\Address{
  Friedrich Leisch\\
  Institut f\"ur Statistik \& Wahrscheinlichkeitstheorie\\
  Technische Universit\"at Wien\\
  Wiedner Hauptstra{\ss}e 8-10/1071\\
  1040 Wien, Austria\\
  E-mail: \email{Friedrich.Leisch@tuwien.ac.at}\\
  URL: \url{http://www.ci.tuwien.ac.at/~leisch/}
}

\usepackage{listings}
\newcommand{\R}{\proglang{R}}

\Abstract{
  FlexMix implements a general framework for fitting
  discrete mixtures of regression models in the \R{} statistical
  computing environment: three variants of the EM algorithm can be
  used for parameter estimation, regressors and responses may be
  multivariate with arbitrary dimension, data may be grouped, e.g., to
  account for multiple observations per individual, the usual formula
  interface of the \proglang{S} language is used for convenient model
  specification, and a modular concept of driver functions allows to
  interface many different types of regression models. Existing
  drivers implement mixtures of standard linear models, generalized
  linear models and model-based clustering. FlexMix provides the
  E-step and all data handling, while the M-step can be supplied by
  the user to easily define new models.
}

\Keywords{\proglang{R}, finite mixture models, model based clustering, latent
  class regression}
\Plainkeywords{R, finite mixture models, model based clustering, latent
  class regression}

\Volume{11}
\Issue{8}
\Month{October}
\Year{2004}
\Submitdate{2004-04-19}
\Acceptdate{2004-10-18}

%%\VignetteIndexEntry{FlexMix: A General Framework for Finite Mixture Models and Latent Class Regression in R}}
%%\VignetteDepends{flexmix}
%%\VignetteKeywords{R, finite mixture models, model based clustering, latent class regression}
%%\VignettePackage{flexmix}

\usepackage{/home/Leisch/work/R/build-patched/share/texmf/Sweave}
\begin{document}

\section{Introduction}
\label{sec:introduction}

Finite mixture models have been used for more than 100 years, but have
seen a real boost in popularity over the last decade due to the
tremendous increase in available computing power. The areas of
application of mixture models range from biology and medicine to
physics, economics and marketing.  On the one hand these models can be
applied to data where observations originate from various groups and
the group affiliations are not known, and on the other hand to provide
approximations for multi-modal distributions
\citep{flb:Everitt+Hand:1981,flb:Titterington+Smith+Makov:1985,flb:McLachlan+Peel:2000}.

In the 1990s finite mixture models have been extended by mixing
standard linear regression models as well as generalized linear models
\citep{flm:Wedel+DeSarbo:1995}. An important area of application of
mixture models is market segmentation \citep{flb:Wedel+Kamakura:2001},
where finite mixture models replace more traditional cluster analysis
and cluster-wise regression techniques as state of the art.  Finite
mixture models with a fixed number of components are usually estimated
with the expectation-maximization (EM) algorithm within a maximum
likelihood framework \citep{flm:Dempster+Laird+Rubin:1977} and with
MCMC sampling \citep{flm:Diebolt+Robert:1994} within a Bayesian
framework.

The \R{} environment for statistical computing \citep{flm:r-core:2004}
features several packages for finite mixture models, including
\pkg{mclust} for mixtures of multivariate Gaussian distributions
\citep{fla:Fraley+Raftery:2002,fla:Fraley+Raftery:2002a}, \pkg{fpc}
for mixtures of linear regression models \citep{flm:Hennig:2000} and
\pkg{mmlcr} for mixed-mode latent class regression
\citep{flm:Buyske:2003}.

There are three main reasons why we have chosen to write yet another
software package for EM estimation of mixture models:
\begin{itemize}
 \item The existing implementations did not cover all cases we needed
  for our own research (mainly marketing applications).
 \item While all \R{} packages mentioned above are open source and hence
  can be extended by the user by modifying the source code, we wanted
  an implementation where extensibility is a main design principle to
  enable rapid prototyping of new mixture models.
 \item We include a sampling-based variant of the EM-algorithm for
  models where weighted maximum likelihood estimation is not
  available. FlexMix has a clean interface between E- and M-step such
  that variations of both are easy to combine.
\end{itemize}

This paper is organized as follows: First we introduce the
mathematical models for latent class regression in
Section~\ref{sec:latent-class-regr} and shortly discuss parameter
estimation and identifiability.  Section~\ref{sec:using-flexmix}
demonstrates how to use FlexMix to fit models with the standard driver
for generalized linear models. Finally,
Section~\ref{sec:extending-flexmix} shows how to extend FlexMix by
writing new drivers using the well-known model-based clustering
procedure as an example.

All computations and graphics in this paper have been done with
\pkg{flexmix} version 1.0-0 and \R{} version 2.0.0 using Sweave
\citep{e1071-papers:Leisch:2002}. The newest release version of
\pkg{flexmix} is always available from the Comprehensive R Archive
Network at \url{http://cran.R-project.org}, an up-to-date version of
this paper is contained as a package vignette.

\section{Latent class regression}
\label{sec:latent-class-regr}


Consider finite mixture models with $K$ components of form
\begin{equation}\label{eq:1}
  h(y|x,\psi) = \sum_{k = 1}^K \pi_k
  f(y|x,\theta_k) 
\end{equation}
\begin{displaymath}
  \pi_k \geq 0, \quad
  \sum_{k = 1}^K \pi_k = 1
\end{displaymath}
where $y$ is a (possibly multivariate) dependent variable with
conditional density $h$, $x$ is a vector of independent variables,
$\pi_k$ is the prior probability of component $k$, $\theta_k$ is the
component specific parameter vector for the density function $f$, and
$\psi=(\pi_1,,\ldots,\pi_K,\theta_1',\ldots,\theta_K')'$ is the vector of
all parameters.

If $f$ is a univariate normal density with component-specific mean
$\beta_k'x$ and variance $\sigma^2_k$, we have $\theta_k = (\beta_k',
\sigma_k^2)'$ and Equation~(\ref{eq:1}) describes a mixture of
standard linear regression models, also called \emph{latent class
  regression} or \emph{cluster-wise regression}
\citep{flm:DeSarbo+Cron:1988}.  If $f$ is a member of the exponential
family, we get a mixture of generalized linear models
\citep{flm:Wedel+DeSarbo:1995}, known as \emph{GLIMMIX} models in the
marketing literature \citep{flb:Wedel+Kamakura:2001}.  For
multivariate normal $f$ and $x\equiv1$ we get a mixture of Gaussians
without a regression part, also known as \emph{model-based
  clustering}.

The posterior probability that observation $(x,y)$ belongs to class
$j$ is given by
\begin{equation}\label{eq:3}
  \Prob(j|x, y, \psi) =
  \frac{\pi_j f(y | x, \theta_j)}{\sum_k \pi_k f(y | x, \theta_k)}
\end{equation}
The posterior probabilities can be used to segment data by assigning
each observation to the class with maximum posterior probability.  In
the following we will refer to $f(\cdot|\cdot, \theta_k)$ as
\emph{mixture components} or \emph{classes}, and the groups in the
data induced by these components as \emph{clusters}.

\subsection{Parameter estimation}
\label{sec:parameter-estimation}

The log-likelihood of a sample of $N$ observations
$\{(x_1,y_1),\ldots,(x_N,y_N)\}$ is given by
\begin{equation}\label{eq:4}
  \log L = \sum_{n=1}^N \log h(y_n|x_n,\psi) =
  \sum_{n=1}^N \log\left(\sum_{k = 1}^K \pi_kf(y_n|x_n,\theta_k) \right)
\end{equation}
and can usually not be maximized directly. The most popular method for
maximum likelihood estimation of the parameter vector $\psi$ is the
iterative EM algorithm \citep{flm:Dempster+Laird+Rubin:1977}: 
\begin{description}
 \item[Estimate] the posterior class probabilities for each observation
  \begin{displaymath}
    \hat p_{nk} = \Prob(k|x_n, y_n, \hat \psi)
  \end{displaymath}
  using Equation~(\ref{eq:3}) and derive the prior class probabilities as
  \begin{displaymath}
    \hat\pi_k = \frac1N \sum_{n=1}^N \hat p_{nk}
  \end{displaymath}

 \item[Maximize] the log-likelihood for each component separately using the
  posterior probabilities as weights
  \begin{equation}\label{eq:2}
    \max_{\theta_k} \sum_{n=1}^N \hat p_{nk} \log f(y_n | x_n, \theta_k)  
  \end{equation}
\end{description}
The E- and M-steps are repeated until the likelihood improvement falls
under a pre-specified threshold or a maximum number of iterations is
reached.

The EM algorithm cannot be used for mixture models only, but rather
provides a general framework for fitting models on incomplete data.
Suppose we augment each observation $(x_n,y_n)$ with an unobserved
multinomial variable $z_n = (z_{n1},\ldots,z_{nK})$, where $z_{nk}=1$
if $(x_n,y_n)$ belongs to class $k$ and $z_{nk}=0$ otherwise. The EM
algorithm can be shown to maximize the likelihood on the ``complete
data'' $(x_n,y_n,z_n)$;  the $z_n$ encode the missing class
information. If the $z_n$ were known, maximum likelihood estimation of
all parameters would be easy, as we could separate the data set into
the $K$ classes and estimate the parameters $\theta_k$ for each class
independently from the other classes.

If the weighted likelihood estimation in Equation~(\ref{eq:2}) is
infeasible for analytical, computational, or other reasons, then we
have to resort to approximations of the true EM procedure by assigning
the observations to disjoint classes and do unweighted estimation
within the groups:
\begin{displaymath}
  \max_{\theta_k} \sum_{n: z_{nk=1}} \log f(y_n | x_n, \theta_k)  
\end{displaymath}
This corresponds to allow only 0 and 1 as weights.

Possible ways of assigning the data into the $K$ classes are
\begin{itemize}
 \item \textbf{hard} \label{hard} assignment to the class with maximum
  posterior probability $p_{nk}$, the resulting procedure is called
  maximizing the \emph{classification likelihood} by
  \cite{fla:Fraley+Raftery:2002}. Another idea is to do
 \item \textbf{random} assignment to classes with probabilities
  $p_{nk}$, which is similar to the sampling techniques used in
  Bayesian estimation (although for the $z_n$ only).
\end{itemize}

Well known limitations of the EM algorithm include that convergence
can be slow and is to a local maximum of the likelihood surface only.
There can also be numerical instabilities at the margin of parameter
space, and if a component gets to contain only a few observations
during the iterations, parameter estimation in the respective
component may be problematic.  E.g., the likelihood of Gaussians
increases without bounds for $\sigma^2\to 0$. As a result, numerous
variations of the basic EM algorithm described above exist, most of
them exploiting features of special cases for $f$.

\subsection{Identifiability}
\label{sec:identifiability}

An open question is still identifiability of many mixture models. A
comprehensive overview of this topic is beyond the scope of this
paper, however, users of mixture models should be aware of the
problem:
\begin{description}
 \item[Relabelling of components:] Mixture models are only
  identifiable up to a permutation of the component labels. For
  EM-based approaches this only affects interpretation of results, but
  is no problem for parameter estimation itself.
  
 \item[Overfitting:] If a component is empty or two or more components
  have the same parameters, the data generating process can be
  represented by a smaller model with fewer components.  This kind of
  unidentifiability can be avoided by requiring that the prior weights
  $\pi_k$ are not equal to zero and that the component specific
  parameters are different.
  
 \item[Generic unidentifiability:] It has been shown that mixtures of
  univariate normal, gamma, exponential, Cauchy and Poisson
  distributions are identifiable, while mixtures of discrete or
  continuous uniform distributions are not identifiable. A special
  case is the class of mixtures of binomial and multinomial
  distributions which are only identifiable if the number of
  components is limited with respect to, e.g., the number of
  observations per person. See \cite{flb:Everitt+Hand:1981},
  \cite{flb:Titterington+Smith+Makov:1985},
  \cite{e1071-theses:Grun:2002} and references therein for details.
\end{description}

FlexMix tries to avoid overfitting because of vanishing prior
probabilities by automatically removing components where the prior
$\pi_k$ falls below a user-specified threshold. Automated diagnostics
for generic identifiability are currently under investigation.
Relabelling of components is in some cases more of a nuisance than a
real problem (``component 2 of the first run may be component 3 in the
second run''), more serious are interactions of component relabelling
and categorical predictor variables, see
\cite{e1071-papers:Grun+Leisch:2004} for a discussion and how
bootstrapping can be used to assess identifiability of mixture models.


\section{Using FlexMix}
\label{sec:using-flexmix}




The standard M-step \texttt{FLXglm()} of FlexMix is an interface to
R's generalized linear modelling facilities (the \texttt{glm()}
function).  As a simple example we use artificial data with two latent
classes of size 100 each:
\begin{center}
  \begin{tabular}{ll}
    Class~1: & $ y = 5x+\epsilon$\\
    Class~2: & $ y = 15+10x-x^2+\epsilon$\\
  \end{tabular}
\end{center}
with $\epsilon\sim N(0,9)$ and prior class probabilities
$\pi_1=\pi_2=0.5$, see the left panel of Figure~\ref{fig:npreg}. 

We can fit this model in \R{} using the commands
\begin{Schunk}
\begin{Sinput}
> library(flexmix)
> data(NPreg)
> m1 = flexmix(yn ~ x + I(x^2), data = NPreg, k = 2)
> m1
\end{Sinput}
\begin{Soutput}
Call:
flexmix(formula = yn ~ x + I(x^2), data = NPreg, 
    k = 2)

Cluster sizes:
  1   2 
100 100 

convergence after 15 iterations
\end{Soutput}
\end{Schunk}
and get a first look at the estimated parameters of mixture component~1 by 
\begin{Schunk}
\begin{Sinput}
> parameters(m1, component = 1)
\end{Sinput}
\begin{Soutput}
$coef
(Intercept)           x      I(x^2) 
-0.20989331  4.81782414  0.03615728 

$sigma
[1] 3.476362
\end{Soutput}
\end{Schunk}
and
\begin{Schunk}
\begin{Sinput}
> parameters(m1, component = 2)
\end{Sinput}
\begin{Soutput}
$coef
(Intercept)           x      I(x^2) 
 14.7168295   9.8466698  -0.9683534 

$sigma
[1] 3.479809
\end{Soutput}
\end{Schunk}
for component~2. The paramter estimates of both components are close
to the true values. A cross-tabulation of true classes and cluster
memberships can be obtained by
\begin{Schunk}
\begin{Sinput}
> table(NPreg$class, m1@cluster)
\end{Sinput}
\begin{Soutput}
    1  2 
  1 95  5
  2  5 95
\end{Soutput}
\end{Schunk}
The summary method
\begin{Schunk}
\begin{Sinput}
> summary(m1)
\end{Sinput}
\begin{Soutput}
Call:
flexmix(formula = yn ~ x + I(x^2), data = NPreg, 
    k = 2)

       prior size post>0 ratio
Comp.1 0.494  100    145 0.690
Comp.2 0.506  100    141 0.709

`log Lik.' -642.5453 (df=9)
AIC: 1303.091   BIC: 1332.775 
\end{Soutput}
\end{Schunk}
gives the estimated prior probabilities $\hat\pi_k$, the
number of observations assigned to the corresponding clusters, the
number of observations where $p_{nk}>\delta$ (with a default of
$\delta=10^{-4}$), and the ratio of the latter two numbers. For
well-seperated components, a large proportion of observations with
non-vanishing posteriors $p_{nk}$ should also be assigned to the
corresponding cluster, giving a ratio close to 1. For our example data the
ratios of both components are approximately 0.7, indicating the
overlap of the classes at the cross-section of line and parabola.



\begin{figure}[htbp]
  \centering
\includegraphics{flexmix-intro-007}
  \caption{Standard regression example (left) and Poisson regression (right).}
  \label{fig:npreg}
\end{figure}

Histograms or rootograms of the posterior class probabilities can be
used to visually assess the cluster structure
\citep{fla:Tantrum+Murua+Stuetzle:2003}, this is now the default plot
method for \texttt{"flexmix"} objects
\citep{e1071-papers:Leisch:2004}. Rootograms are very similar to
histograms, the only difference is that the height of the bars
correspond to square roots of counts rather than the counts
themselves, hence low counts are more visible and peaks less
emphasized.

\begin{figure}[htbp]
  \centering
\includegraphics{flexmix-intro-008}
\caption{The plot method for \texttt{"flexmix"} objects, here obtained
  by \texttt{plot(m1)}, shows rootograms of the posterior class
  probabilities.}
  \label{fig:root1}
\end{figure}

Usually in each component a lot of observations have posteriors close
to zero, resulting in a high count for the corresponing bin in the
rootogram which obscures the information in the other bins. To avoid
this problem, all probabilities with a posterior below a threshold are
ignored (we again use $10^{-4}$).  A peak at probability 1 indicates that a
mixture component is well seperated from the other components, while
no peak at 1 and/or significant mass in the middle of the unit
interval indicates overlap with other components. In our simple
example the components are medium well separated, see
Figure~\ref{fig:root1}. 

Tests for significance of regression coefficients can be obtained by
\begin{Schunk}
\begin{Sinput}
> rm1 = refit(m1)
> summary(rm1)
\end{Sinput}
\begin{Soutput}
Call:
refit(m1)

Component 1 :
             Estimate Std. Error t value Pr(>|t|)
(Intercept) -0.208996   0.673900 -0.3101   0.7568
x            4.817015   0.327447 14.7108   <2e-16
I(x^2)       0.036233   0.032545  1.1133   0.2669

-------------
Component 2 :
             Estimate Std. Error t value  Pr(>|t|)
(Intercept) 14.717541   0.890843  16.521 < 2.2e-16
x            9.846148   0.390385  25.222 < 2.2e-16
I(x^2)      -0.968304   0.036951 -26.205 < 2.2e-16
\end{Soutput}
\end{Schunk}
Function \texttt{refit()} fits weighted generalized linear models to each
component using the standard \R{} function \texttt{glm()} and the
posterior probabilities as weights, see \texttt{help("refit")} for
details.

The data set \texttt{NPreg} also includes a response from a
generalized linear model with a Poisson distribution and exponential
link function. The two classes of size 100 each have
parameters
\begin{center}
  \begin{tabular}{ll}
    Class~1: & $ \mu_1 = 2 - 0.2x$\\
    Class~2: & $ \mu_2 = 1 + 0.1x$\\
  \end{tabular}
\end{center}
and given $x$ the response $y$ in group $k$ has a Poisson distribution
with mean $e^{\mu_k}$, see the right panel of Figure~\ref{fig:npreg}.
The model can be estimated using

\begin{Schunk}
\begin{Sinput}
> m2 = flexmix(yp ~ x, data = NPreg, k = 2, model = FLXglm(family = "poisson"))
> summary(m2)
\end{Sinput}
\begin{Soutput}
Call:
flexmix(formula = yp ~ x, data = NPreg, k = 2, model = FLXglm(family = "poisson"))

       prior size post>0 ratio
Comp.1 0.532  112    197 0.569
Comp.2 0.468   88    200 0.440

`log Lik.' -440.6425 (df=5)
AIC: 891.285   BIC: 907.7767 
\end{Soutput}
\end{Schunk}

\begin{figure}[htbp]
  \centering
\includegraphics{flexmix-intro-011}
  \caption{\texttt{plot(m2)}}
  \label{fig:root2}
\end{figure}

Both the summary table and the rootograms in Figure~\ref{fig:root2}
clearly show that the clusters of the Poisson response have much more
overlap. For our simple low-dimensional example data the overlap of
the classes is obvious by looking at scatterplots of the data. For
data in higher dimensions this is not an option. The rootograms and
summary tables for \texttt{"flexmix"} objects work off the densities
or posterior probabilities of the observations and thus do not depend
on the dimensionality of the input space. While we use simple
2-dimensional examples to demonstrate the techniques, they can easily
be used on high-dimensional data sets or models with complicated
covariate structures.


\subsection{Multiple independent responses}
\label{sec:mult-indep-resp}


If the response $y=(y_1,\ldots,y_D)'$ is $D$-dimensional and the $y_d$
are mutually independent the mixture density in Equation~(\ref{eq:1})
can be written as
\begin{eqnarray*}
  h(y|x,\psi) &=& \sum_{k = 1}^K \pi_k
  f(y|x,\theta_k)\\
  &=&  \sum_{k = 1}^K \pi_k
  \prod_{d=1}^D f_d(y|x,\theta_{kd})
\end{eqnarray*}
To specify such models in FlexMix we pass it a list of models, where
each list element corresponds to one $f_d$, and each can have a
different set of dependent and independent variables.  To use the
Gaussian and Poisson responses of data \texttt{NPreg} simultaneously,
we use the model specification
\begin{Sinput}
> m3 = flexmix(~x, data=NPreg, k=2,
+              model=list(FLXglm(yn~.+I(x^2)), 
+                         FLXglm(yp~., family="poisson")))  
\end{Sinput}

Note that now three model formulas are involved: An overall formula as
first argument to function \texttt{flexmix()} and one formula per
response. The latter ones are interpreted relative to the overall
formula such that common predictors have to be specified only once,
see \texttt{help("update.formula")} for details on the syntax. The
basic principle is that the dots get replaced by the respective terms
from the overall formula. The rootograms show that the posteriors of
the two-response model are shifted towards 0 and 1 (compared with
either of the two univariate models), the clusters are now
well-separated.

\begin{figure}[htbp]
  \centering
\includegraphics{flexmix-intro-013}
  \caption{\texttt{plot(m3)}}
  \label{fig:root3}
\end{figure}



\subsection{Repeated measurements}
\label{sec:repe-meas}

If the data are repeated measurements on $M$ individuals, and we have
$N_m$ observations from individual $m$, then the log-likelihood in
Equation~(\ref{eq:4}) can be written as
\begin{displaymath}
  \log L = \sum_{m=1}^M \sum_{n=1}^{N_m} \log h(y_{mn}|x_{mn},\psi),
  \qquad \sum_{m=1}^M N_m = N
\end{displaymath}
and the posterior probability that individual $m$ belongs to class
$j$ is given by
\begin{displaymath}
  \Prob(j|m) = \frac{\pi_j \prod_{n=1}^{N_m} f(y_{mn} | x_{mn}, \theta_j)}{\sum_k \pi_k \prod_{n=1}^{N_m} f(y_{mn} | x_{mn}, \theta_k)}
\end{displaymath}
where $(x_{mn}, y_{mn})$ is the $n$-th observation from individual $m$.
As an example, assume that the data in \texttt{NPreg} are not 200
independent observations, but 4 measurements each from 50 persons such
that $\forall m: N_m=4$.  Column \texttt{id2} of the data frame
encodes such a grouping and can easily be used in FlexMix:

\begin{Schunk}
\begin{Sinput}
> m4 = flexmix(yn ~ x + I(x^2) | id2, data = NPreg, 
+     k = 2)
> summary(m4)
\end{Sinput}
\begin{Soutput}
Call:
flexmix(formula = yn ~ x + I(x^2) | id2, data = NPreg, 
    k = 2)

       prior size post>0 ratio
Comp.1   0.5  100    100     1
Comp.2   0.5  100    100     1

`log Lik.' -2246.26 (df=9)
AIC: 4510.52   BIC: 4540.205 
\end{Soutput}
\end{Schunk}
Note that convergence of the EM algorithm is much faster with
grouping and the two clusters are now perfectly separated.



\subsection{Control of the EM algorithm}
\label{sec:control-em-algorithm}

Details of the EM algorithm can be tuned using the \texttt{control}
argument of function \texttt{flexmix()}. E.g., to use a maximum number
of 15 iterations, report the log-likelihood at every 3rd step and use hard
assignment of observations to clusters (cf. page~\pageref{hard}) the
call is

\begin{Schunk}
\begin{Sinput}
> m5 = flexmix(yn ~ x + I(x^2), data = NPreg, k = 2, 
+     control = list(iter.max = 15, verbose = 3, classify = "hard"))
\end{Sinput}
\begin{Soutput}
Classification: hard 
   3 Log-likelihood :    -718.2839 
   6 Log-likelihood :    -712.2280 
   9 Log-likelihood :    -711.1906 
  12 Log-likelihood :    -711.0684 
  13 Log-likelihood :    -711.0680 
converged
\end{Soutput}
\end{Schunk}

Another control parameter (\texttt{minprior}, see below for an
example) is the minimum prior probability components are enforced to
have, components falling below this threshold (the current default is
0.05) are removed during EM iteration to avoid numerical instabilities
for components containing only a few observations. Using a minimum
prior of 0 disables component removal.



\subsection{Automated model search}

In real applications the number of components is unknown and has to be
estimated.  Tuning the minimum prior parameter allows for 
simplistic model selection, which works surprisingly well in some situations:
\begin{Schunk}
\begin{Sinput}
> m6 = flexmix(yp ~ x + I(x^2), data = NPreg, k = 4, 
+     control = list(minprior = 0.2))
> m6
\end{Sinput}
\begin{Soutput}
Call:
flexmix(formula = yp ~ x + I(x^2), data = NPreg, 
    k = 4, control = list(minprior = 0.2))

Cluster sizes:
  1   2 
 79 121 

convergence after 142 iterations
\end{Soutput}
\end{Schunk}
Although we started with four components, the algorithm converged at
the correct two component solution.

A better approach is to fit models with an increasing number of
components and compare them using AIC or BIC. As the EM algorithm
converges only to the next local maximum of the likelihood, it should
be run repeatedly using different starting values. The function
\texttt{stepFlexmix()} can be used to repeatedly fit models, e.g.,
\begin{Schunk}
\begin{Sinput}
> m7 = stepFlexmix(yp ~ x + I(x^2), data = NPreg, control = list(verbose = 0), 
+     K = 1:5, nrep = 5)
\end{Sinput}
\begin{Soutput}
1 : * * * * *
2 : * * * * *
3 : * * * * *
4 : * * * * *
5 : * * * * *
\end{Soutput}
\end{Schunk}
runs \texttt{flexmix()} 5 times for $K=1,2,\ldots,5$ components,
totalling in 25 runs. It returns a list with the best solution found
for each number of components, each list element is simply an object
of class \texttt{"flexmix"}. To find the best model we can use
\begin{Schunk}
\begin{Sinput}
> sapply(m7, BIC)
\end{Sinput}
\begin{Soutput}
       1        2        3        4        5 
946.7477 925.9972 942.1553 960.0626 960.9347 
\end{Soutput}
\end{Schunk}
and choose the number of components minimizing the BIC.




\section{Extending FlexMix}
\label{sec:extending-flexmix}

One of the main design principles of FlexMix was extensibility, users
can provide their own M-step for rapid prototyping of new mixture
models. FlexMix was written using S4 classes and methods
\citep{flb:Chambers:1998} as implemented in \R{} package
\pkg{methods}. 

The central classes for writing M-steps are \texttt{"FLXmodel"} and
\texttt{"FLXcomponent"}. Class \texttt{"FLXmodel"} specifies how the
model is fitted using the following slots:
\begin{description}
 \item[fit:] A \texttt{function(x,y,w)} returning an object of class
  \texttt{"FLXcomponent"}.
 \item[weighted:] Logical, specifies if the model may be fitted using
  weighted likelihoods. If \texttt{FALSE}, only hard and random
  classification are allowed (and hard classification becomes the
  default). 
 \item[formula:] Formula relative to the overall model formula, default is
  \verb|.~.|
 \item[name:] A character string describing the model, this is only
  used for print output. 
\end{description}
The remaining slots of class \texttt{"FLXmodel"} are used internally
by FlexMix to hold data, etc. and omitted here, because they are not
needed to write an M-step driver. The most important slot doing all
the work is \texttt{fit} holding a function performing the maximum
likelihood estimation described in Equation~(\ref{eq:2}). The
\texttt{fit()} function returns an object of class \texttt{"FLXcomponent"}
which holds a fitted component using the slots:
\begin{description}
 \item[logLik:] A \texttt{function(x,y)} returning the
  log-likelihood for observations in matrices \texttt{x} and \texttt{y}.
 \item[predict:] A \texttt{function(x)} predicting \texttt{y} given
  \texttt{x}.
 \item[df:] The degrees of freedom used by the component, i.e., the
  number of estimated parameters.
 \item[parameters:] An optional list containing model parameters.
\end{description}
In a nutshell class \texttt{"FLXmodel"} describes an \emph{unfitted}
model, whereas class \texttt{"FLXcomponent"} holds a \emph{fitted}
model.

\lstset{frame=trbl,basicstyle=\small\tt,stepnumber=5,numbers=left}
                                                                               
\begin{figure}[tb]
  \centering
  \begin{minipage}{0.94\textwidth}
    \lstinputlisting{mymclust.R}
  \end{minipage}
  \caption{M-step for model-based clustering: \texttt{mymclust} is a
    simplified version of the standard FlexMix driver \texttt{FLXmclust}.}
  \label{fig:mymclust.R}
\end{figure}


\subsection{Writing an M-step driver}
\label{sec:writing-an-m}

Figure~\ref{fig:mymclust.R} shows an example driver for model-based
clustering. We use function \texttt{dmvnorm()} from package
\pkg{mvtnorm} for calculation of multivariate Gaussian densities.
In line~5 we create a new \texttt{"FLXmodel"} object named \texttt{retval},
which is also the return value of the driver.  All drivers should take
a formula as their first argument, this formula is directly passed on to
\texttt{retval}. In most cases authors of new FlexMix drivers need not
worry about formula parsing etc., this is done by \texttt{flexmix}
itself. In addition we have to declare whether the driver can do
weighted ML estimation (\texttt{weighted=TRUE}) and give a name to our
model.

The remainder of the driver creates a \texttt{fit()} function, which
takes regressors \texttt{x}, response \texttt{y} and weights
\texttt{w}. For multivariate Gaussians the maximum likelihood
estimates correspond to mean and covariance matrix, the standard R
function \texttt{cov.wt()} returns a list containing estimates of the
weighted covariance matrix and the mean for given data. Our simple
example performs clustering without a regression part, hence $x$ is
ignored. If \texttt{y} has $D$ columns, we estimate $D$ parameters for
the mean and $D(D-1)/2$ parameters for the covariance matrix, giving a
total of $(3D+D^2)/2$ parameters (line~11).  As an additional feature
we allow the user to specify whether the covariance matrix is assumed
to be diagonal or a full matrix. For \texttt{diagonal=TRUE} we use
only the main diagonal of the covariance matrix (line~14) and the
number of parameters is reduced to $2D$.

In addition to parameter estimates, \texttt{flexmix()} needs a function
calculating the log-likelihood of given data $x$ and $y$, which in our
example is the log-density of a multivariate Gaussian. In addition we
have to provide a function predicting $y$ given $x$, in our example
simply the mean of the Gaussian. Finally we create a new
\texttt{"FLXcomponent"} as return value of function \texttt{fit()}.

Note that our internal functions \texttt{fit()}, \texttt{logLik()} and
\texttt{predict()} take only \texttt{x}, \texttt{y} and \texttt{w} as
arguments, but none of the model-specific parameters like means and
covariances, although they use them of course. \R{} uses \emph{lexical
  scoping} rules for finding free variables
\citep{fla:Gentleman+Ihaka:2000}, hence it searches for them first in
the environment where a function is defined. E.g., the \texttt{fit()}
function uses the variable \texttt{diagonal} in line~13, and finds it
in the environment where the function itself was defined, which is the
body of function \texttt{mymclust()}. Function \texttt{logLik()} uses
the list \texttt{para} in line~19, and uses the one found in the
body of \texttt{fit()}.

Function \texttt{flexmix()} on the other hand never sees the model
parameters, all it uses are function calls of form \texttt{fit(x,y,w)}
or \texttt{logLik(x,y)}, which are exactly the same for all kinds of
mixture models. In fact, it would not be necessary to even store the
component parameters in the \texttt{"FLXcomponent"} object, they are
there only for convenience such that users can easily extract and use
them after \texttt{flexmix()} has finished. Lexical scope allows to
write clean interfaces in a very elegant way, the driver abstracts all
model details from the FlexMix main engine.



\subsection{Example: Using the driver}
\label{sec:example:-model-based}




As a simple example we use the four 2-dimensional Gaussian clusters
from data set \texttt{Nclus}. Fitting a wrong model with diagonal
covariance matrix is done by
\begin{Schunk}
\begin{Sinput}
> data(Nclus)
> m1 = flexmix(Nclus ~ 1, k = 4, model = mymclust())
> summary(m1)
\end{Sinput}
\begin{Soutput}
Call:
flexmix(formula = Nclus ~ 1, k = 4, model = mymclust())

       prior size post>0 ratio
Comp.1 0.159   92    165 0.558
Comp.2 0.269  149    174 0.856
Comp.3 0.397  213    488 0.436
Comp.4 0.175   96    172 0.558

`log Lik.' -2447.3 (df=19)
AIC: 4932.6   BIC: 5014.489 
\end{Soutput}
\end{Schunk}

The result can be seen in the left panel of Figure~\ref{fig:ell}, the
result is ``wrong'' because we forced the ellipses to be parallel to
the axes. The overlap between three of the four clusters can also be
inferred from the low ratio statistics in the summary table (around
0.5 for components 1, 3 and 4), while the much better separated upper
left cluster has a much higher ratio of 0.85. Using the correct model
with a full covariance matrix can be done by setting
\texttt{diagonal=FALSE} in the call to our driver \texttt{mymclust()}:
\begin{Schunk}
\begin{Sinput}
> m2 = flexmix(Nclus ~ 1, k = 4, model = mymclust(diagonal = FALSE))
> summary(m2)
\end{Sinput}
\begin{Soutput}
Call:
flexmix(formula = Nclus ~ 1, k = 4, model = mymclust(diagonal = FALSE))

       prior size post>0 ratio
Comp.1 0.177   96    132 0.727
Comp.2 0.368  204    247 0.826
Comp.3 0.272  150    176 0.852
Comp.4 0.182  100    112 0.893

`log Lik.' -2223.677 (df=23)
AIC: 4493.355   BIC: 4592.483 
\end{Soutput}
\end{Schunk}

\begin{figure}[htbp]
  \centering
\includegraphics{flexmix-intro-022}
  \caption{Fitting a mixture model with diagonal covariance matrix (left) and full covariance matrix (right).}
  \label{fig:ell}
\end{figure}


\section{Summary and outlook}
\label{sec:summary}

The primary goal of FlexMix is extensibility, this makes the package
ideal for rapid development of new mixture models. There is no intent
to replace packages implementing more specialized mixture models like
\pkg{mclust} for mixtures of Gaussians, FlexMix should rather be
seen as a complement to those. By interfacing R's facilities for
generalized linear models, FlexMix allows the user to estimate complex
latent class regression models.

Using lexical scope to resolve model-specific parameters hides all
model details from the programming interface, FlexMix can in principle
fit almost arbitrary finite mixture models for which the EM algorithm
is applicable. The downside of this is that FlexMix can in principle
fit almost arbitrary finite mixture models, even models where no
proper theoretical results for model identification etc.\ are
available.

We are currently working on a toolset for diagnostic checks on mixture
models to test necessary identifiability conditions for those
cases where results are available. We also want to implement
newer variations of the classic EM algorithm, especially for faster
convergence. Another plan is to have an interactive version of the
rootograms using \texttt{iPlots} \citep{flm:Urbanek+Theus:2003} such
that the user can explore the relations between mixture components,
possibly linked to background variables. Other planned extensions
include covariates for the prior probabilities and to allow to mix
different distributions for components, e.g., to include a Poisson
point process for background noise.

\section*{Acknowledgments}

This research was supported by the Austrian Science Foundation (FWF)
under grant SFB\#010 (`Adaptive Information Systems and Modeling in
Economics and Management Science').

\bibliography{fls,fla,flb,flm,e1071-theses,e1071-papers}

\end{document}


%%% Local Variables: 
%%% mode: latex
%%% TeX-master: t
%%% End: 
